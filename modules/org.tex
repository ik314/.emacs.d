
\section{org-mode}
\label{sec:orge222a5f}
I would be lying if I said emacs was a good text editor out of the
box. It's great when you configure it, sure, but why the hell would
you bother doing that? Answer: Org mode. Look it up.

\begin{verbatim}
(use-package org :ensure t
  :init
  ;; Proper code blocks
  (setq org-src-fontify-natively t)
  (setq org-src-tab-acts-natively t)
  ;; Babel languages
  (org-babel-do-load-languages
   'org-babel-load-languages
   '((python  . t)
     (shell   . t)
     (C       . t)
     (fortran . t)
     (haskell . t)
     (awk     . t)
     (js      . t)
     (R       . t)
     (octave  . t)
     (matlab  . t)
     (perl    . t)
     (gnuplot . t)
     (latex   . t)
     (emacs-lisp . t)))
  ;; Agenda
  (setq org-log-done t)
  ;; Encoding
  (setq org-export-coding-system 'utf-8)
  (prefer-coding-system 'utf-8)
  (set-charset-priority 'unicode)
  (setq default-process-coding-system '(utf-8-unix . utf-8-unix))
  ;; Don't allow editing of folded regions
  (setq org-catch-invisible-edits 'error)
  ;; Start agenda on Monday
  (setq org-agenda-start-on-weekday 1)
  ;; Enable indentation view, does not effect file
  (setq org-startup-indented t)
  ;; Attachments
  (setq org-id-method (quote uuidgen))
  (setq org-attach-directory "attach/")
  :bind
  (("\C-ca" . org-agenda)
   ("\C-cl" . org-store-link)
   ("\C-cc" . org-capture))
  )
:config
  (add-hook 'org-mode-hook (lambda () (org-bullets-mode 1)))

(add-hook 'org-mode-hook #'(lambda ()
			     ;; make the lines in the buffer wrap around the edges of the screen.
			     ;; to press C-c q  or fill-paragraph ever again!
			     (visual-line-mode)
			     (org-indent-mode)))

(add-hook 'org-mode-hook
	   (lambda ()
	   (define-key evil-normal-state-map (kbd "TAB") 'org-cycle)))

(setq company-global-modes '(not org-mode))

(setq org-format-latex-options (plist-put org-format-latex-options :scale 1.5))
\end{verbatim}
\subsection{org-roam!}
\label{sec:org2c7e232}
Developing an ecosystem, a second brain.
\begin{verbatim}
(use-package org-roam
    :ensure t
    :custom
    (org-roam-directory (file-truename "/home/ik/org-roam/"))
    :bind (("C-c n l" . org-roam-buffer-toggle)
	   ("C-c n f" . org-roam-node-find)
	   ("C-c n g" . org-roam-graph)
	   ("C-c n i" . org-roam-node-insert)
	   ("C-c n c" . org-roam-capture)
	       ("C-c n #" . org-roam-tag-add)
	   ;; Dailies
	   ("C-c n j" . org-roam-dailies-capture-today))
    :config

    (org-roam-setup)
    ;; If using org-roam-protocol
    (require 'org-roam-protocol))
    (setq org-roam-v2-ack t)
\end{verbatim}

Also let us include the \texttt{org-roam-ui} package, that uses \href{./websocket.org}{websocket},
\begin{verbatim}
;; also requires simple-httpd
(use-package simple-httpd
  :ensure t)
;; currently not on MEPLA yet so manually install located at .emacs.d/private
(require 'websocket)
(add-to-list 'load-path "~/.emacs.d/private/org-roam-ui")
(load-library "org-roam-ui")
\end{verbatim}

I also want to be able to have firm referencing that allows for back propagating through texts (particularly if they are digital).
\begin{verbatim}
(use-package helm-bibtex
  :ensure t)


(use-package org-ref
  :ensure t
  :config
  (require 'doi-utils)
  (require 'org-ref-isbn)
  (setq reftex-default-bibliography '("~/biblio/references.bib"))
  (setq org-ref-default-bibliography '("~/biblio/references.bib")
	org-ref-pdf-directory "~/research")

  (setq bibtex-completion-bibliography "~/biblio/references.bib"
	bibtex-completion-library-path "~/research"
	bibtex-completion-notes-path "~/org-roam"))


(use-package org-roam-bibtex
  :ensure t)
\end{verbatim}

\subsection{org-download}
\label{sec:orgd33fe2c}
\begin{verbatim}
(use-package org-download :ensure t
:config
(require 'org-download
(add-hook 'dired-mode-hook 'org-download-enable)))
\end{verbatim}

\subsection{org-bullets}
\label{sec:org227758c}
\begin{verbatim}
(use-package org-bullets :ensure t
:config
(require 'org-bullets))
\end{verbatim}
\subsection{org-cliplink}
\label{sec:orgb16cc0b}
\begin{verbatim}
(use-package org-cliplink :ensure t
)
\end{verbatim}

\subsection{org-agenda (General GTD Organization)}
\label{sec:orgb9c1346}
\begin{verbatim}
  (setq org-agenda-files '("~/gtd/inbox.org"
			  "~/gtd/gtd.org"
			  "~/gtd/tickler.org"))

  (setq org-capture-templates '(("t" "Todo [inbox]" entry
				 (file+headline "~/gtd/inbox.org" "Tasks")
				 "* TODO %i%?")
				("T" "Tickler" entry
				 (file+headline "~/gtd/tickler.org" "Tickler")
				 "* %i%? \n %U")))

  (setq org-refile-targets '(("~/gtd/gtd.org" :maxlevel . 3)
			     ("~/gtd/someday.org" :level . 1)
			     ("~/gtd/tickler.org" :maxlevel . 2)))
; Pressing C-c C-t sets the TODO keyword.
  (setq org-todo-keywords '((sequence "TODO(t)" "WAITING(w)" "|" "DONE(d)" "CANCELLED(c)")))

  (setq org-agenda-custom-commands 
	'(("o" "At the office" tags-todo "@office"
	   ((org-agenda-overriding-header "Office")
	    (org-agenda-skip-function #'my-org-agenda-skip-all-siblings-but-first)))))

  (defun my-org-agenda-skip-all-siblings-but-first ()
    "Skip all but the first non-done entry."
    (let (should-skip-entry)
      (unless (org-current-is-todo)
	(setq should-skip-entry t))
      (save-excursion
	(while (and (not should-skip-entry) (org-goto-sibling t))
	  (when (org-current-is-todo)
	    (setq should-skip-entry t))))
      (when should-skip-entry
	(or (outline-next-heading)
	    (goto-char (point-max))))))

  (defun org-current-is-todo ()
    (string= "TODO" (org-get-todo-state)))
\end{verbatim}

\subsection{Export Settings for \LaTeX{}}
\label{sec:org434bec6}
\begin{verbatim}
(add-to-list 'org-latex-classes
	   '("standard"
		 "\\documentclass{article}
		  \\usepackage[margin=0.8in]{geometry}
		  \\displaymode
		  \\usepackage{parskip}
		  \\usepackage{amsmath} 
		  \\usepackage{import}
		  \\usepackage{xifthen}
		  \\usepackage{pdfpages}
		  \\usepackage{transparent}
		  \\usepackage{hyperref}
		  \\newcommand{\\incfic}[1]{%
		      \\def\svgwidth{\\columnwidth}
		      \\import{./figures/}{#1.pdf_tex}}
		  [NO-DEFAULT-PACKAGES]
		  [NO-PACKAGES]"
		 ("\\section*{%s}" . "\\section*{%s}")
	       ("\\subsection*{%s}" . "\\subsection*{%s}")
	       ("\\subsubsection*{%s}" . "\\subsubsection*{%s}")
	       ("\\paragraph*{%s}" . "\\paragraph*{%s}")
	       ("\\subparagraph*{%s}" . "\\subparagraph*{%s}")))
\end{verbatim}
